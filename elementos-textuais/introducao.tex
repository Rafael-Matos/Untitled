\chapter{Introdução}
\label{}
Devido a um problema de analise climáticas de certas regiões de difícil acesso como no nordeste, para monitorar as condições meteorológicas para buscar entender melhor sobre o funcionamento e uso na agricultura. 

Para levantamento de informações sobre as condições climáticas Como temperatura, umidade, pressão e luminosidade é necessário muita observação, prática ,seguir horários e gerenciamentos das atividades, que podem afeta muito nas tomadas de decisões a longo prazo  e também nesse ramo dependem de boas condições climáticas para ter sucesso como fazer plantação, pequenas construções e ate mesmo saber se esta confortável para os trabalhadores exercerem suas atividades.  

Com base desses problemas eu formulei um projeto de sistemas embarcados para desenvolver uma estação meteorológica justamente para registar os fenômenos climáticos e possibilitar um monitoramento das variáveis climáticas.

A estação funciona através de sensores, a coleta e a medição de dados climáticos e auxiliar quem esta buscando a melhorar o seu cultivo. E com bases nesses dados obtidos através dos sensores mostrar a situação climática em um dispositivo portátil sem fio facilitando o monitoramento das atividades  agrária. A partir dessas considerações visa-se responder a seguinte pergunta: Como podemos analisar e monitorar melhor as situações climáticas em qualquer lugar? 

Sobre esse projeto envolve algumas tecnologias de sistemas embarcados, que tem a vital importância para automação e são bastante utilizado  no nosso cotidiano, sendo mais viável para solucionar problemas a longo prazo. O Embarcado a ser utilizado é o ESP32 que tem bom poder de processamento tendo um microcontrolador com 2 núcleos, comunicação WI-FI,  podendo integrar com vários sensores e atuadores e acompanha com vários recursos que torna o uso em Internet das coisas algo bem interessante. 

Nessa tecnologia faz integração com os sensores, que são os periféricos que captam o ambiente e gera a informação para o sistema como temperatura, umidade em forma de sinal e o mesmo gerencia esses periféricos e faz a interface com o dispositivo de saída que nesse caso as afirmações são mostradas na tela do celular.  

Antigamente o sistema embarcados era utilizado apenas para aplicações especificas utilizando codificação de alto nível que o torna complicado para desenvolvimento de algumas soluções, atualmente conta com diversas unidades de processamento e com bom desempenho que suporta diversos equipamentos facilitando o desenvolvimento de vários projetos.  

Enfim é bem interessante desenvolver uma estação meteorológica para auxilias as pessoas que esta começando o seu ramo de negócio agrário, que necessitam de informações captadas usando uma placa  ESP32 interligando com vários sensores e mostrando as variáveis climáticas pelo aplicativo em uma rede sem fio. 




\section{Motivação}
\label{sec:motivacao}

Na falta de recursos e na dificuldade de tomar decisões em monitorar os recursos agrários.

\section{Objetivos}
\label{sec:objetivos}



\subsection{Objetivo Geral}
\label{sec:objetivo-geral}

Objetivo deste trabalho é desenvolver ao longo de analise e pesquisa, uma aplicação que usa os recursos dos sistemas embarcados.            

\subsection{Objetivos Específicos}
\label{sec:objetivos-especificos}



	\begin{alineas}
		\item Elaborar um estudo de caso sobre a aplicação.
		\item Desenvolver a programação e entender o funcionamento de cada tarefa.
		\item Montar o projeto baseado nas pesquisas, interligando com os sensores necessários.
		\item Realizar uma analise, teste e experimentos sobre o funcionamento do projeto.
		\item Apresentar os resultados e desenvolver a comclusão sobre o projeto.
	\end{alineas}