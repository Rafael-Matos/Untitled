\chapter{Introdução}
\label{}
	Os sistemas computacionais embarcados estão presentes em praticamente todas as atividadeshumanas e, com os baixos custos tecnológicos atuais, tendem a aumentar sua presença no cotidiano das pessoas. Exemplos de tais sistemas são os telefones celulares com máquina fotográfica e agenda, o sistema de controle dos carros e ônibus, os computadores portáteis palm-top, os fornos de microondas com controle de temperatura inteligente, as máquinas de lavar e outros eletrodomésticos. (CARRO e WAGNER; 2003)
O projeto deste tipo de sistema computacional é extremamente complexo, por envolver conceitos até agora pouco analisados pela computação de propósitos gerais. Por exemplo, as questões da portabilidade e do limite de consumo de potência sem perda de desempenho, a baixa disponibilidade de memória, a necessidade de segurança e confiabilidade, a possibilidade de funcionamento em uma rede maior, e o curto tempo de projeto tornam o desenvolvimento de sistemas computacionais embarcados uma área em si (Wolf 2001).
Segundo Barros e Cavalcante, (2010) a indústria eletrônica tem crescido nos últimos anos a uma taxa impressionante e um dos principais motivos para tal crescimento é a incorporação de sistemas eletrônicos numa grande variedade de produtos tais como automóveis, eletrodomésticos e equipamentos de comunicação pessoal. Sistemas de computação estão presentes em todo lugar e não é supresa que anualmente são produzidos milhões de sistemas destinados a computadores pessoais (desktop), estações de trabalho, servidores e computadores de grande porte. O que pode surpreender, no entanto, é que bilhões de sistemas são produzidos anualmente para as mais diferentes propostas; tais sistemas estão embutidos em equipamentos eletrônicos maiores e executam repetidamente uma função específica de forma transparente para o usuário do equipamento. Como resultado da introdução de sistemas eletrônicos em aplicações tradicionais temos produtos mais eficientes, de melhor qualidade e mais baratos. Dentre os componentes eletrônicos mais utilizados temos os componentes digitais que permitem algum tipo de computação tais como microprocessadores e microcontroladores. 
A maioria das funções dos sistemas eletrônicos atuais, em geral, envolvem algum tipo de computação e controle e são realizadas por componentes digitais. Atualmente é uma tendência que sinais analógicos sejam processados como sinais digitais, de forma que componentes para processamento digital são dominantes nos sistemas eletrônicos 




\section{Motivação}
\label{sec:motivacao}



\section{Objetivos}
\label{sec:objetivos}

Este trabalho sei la oque ghlfisagtfiudss.

\subsection{Objetivo Geral}
\label{sec:objetivo-geral}

Integer imperdiet ac magna eu pulvinar. Aliquam erat volutpat. Etiam molestie, nulla a egestas aliquet, velit augue congue metus, et ultricies metus massa vel nibh. Vivamus viverra commodo finibus. Nam elementum convallis accumsan. Quisque tincidunt purus nisl, tincidunt ultricies odio ultrices eu.

\subsection{Objetivos Específicos}
\label{sec:objetivos-especificos}

Lorem ipsum dolor sit amet, consectetur adipiscing elit. Duis scelerisque, velit at facilisis hendrerit, dui eros lacinia metus, non maximus mi tortor ut lectus. Donec hendrerit leo ut consectetur tincidunt. 

	\begin{alineas}
		\item Lorem ipsum dolor sit amet, consectetur adipiscing elit. Nunc dictum sed tortor nec viverra.
		\item Praesent vitae nulla varius, pulvinar quam at, dapibus nisi. Aenean in commodo tellus. Mauris molestie est sed justo malesuada, quis feugiat tellus venenatis.
		\item Praesent quis erat eleifend, lacinia turpis in, tristique tellus. Nunc dictum sed tortor nec viverra.
		\item Mauris facilisis odio eu ornare tempor. Nunc dictum sed tortor nec viverra.
		\item Curabitur convallis odio at eros consequat pretium.
	\end{alineas}